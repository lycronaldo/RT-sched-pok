%
%                              POK header
% 
% The following file is a part of the POK project. Any modification should
% be made according to the POK licence. You CANNOT use this file or a part
% of a file for your own project.
%
% For more information on the POK licence, please see our LICENCE FILE
%
% Please follow the coding guidelines described in doc/CODING_GUIDELINES
%
%                                      Copyright (c) 2007-2009 POK team 
%
% Created by julien on Thu Apr 30 14:07:20 2009 
%


\chapter{Architecture}
   \section{Directories hierarchy}
   The project is organized with a hierarchy of several directories:
   \begin{itemize}
      \item
         \textbf{examples}: sample code that uses pok and libpok. Code of
         examples is mostly generated from AADL models by Ocarina.

      \item
         \textbf{kernel}: code of the kernel that provides time and space
         partitioning services.

      \item
         \textbf{libpok}: code of libpok, the runtime of each partition. 
         It contains libc, POSIX and arinc653 compliant abstraction layers.

      \item
         \textbf{misc}: misc files, such as makefiles, various tools to 
         compile POK and so on.
   \end{itemize}

   \section{"Schyzophrenic" architecture}
   POK can be used as an executive (i.e a kernel that contains different tasks
   but does not provide partitioning functionnalities) or a partitioned
   architecture (a kernel isolates tasks in so-called partitions in terms
   of space and time).

   Moreover, it was designed to support several API and services. But you can
   finely tune the kernel to avoid unused services, reduce memory footprint and
   ease certification/verification efforts.

   Next sections discusses the different architectures that can be used.

      \subsection{Partitioned architecture}
      The partitioned architecture pattern can be used with POK. In that case,
      the kernel will execute several partitions on top of the POK kernel and
      provide time and space partitioning across partitions. 

      \onehugefig{imgs/pok-layers}{The different pok layers}{figure:pok-layers}

      Each partition contains their memory allocators, their runtime and
      ressources (the so-called \textit{libpok} part). Partitions can have
      different scheduling algorithms to schedule their tasks.

      In that case, the kernel provides communication isolation across
      partitions as well as space isolation (each partition has its own memory
      segment).

      The overall architecture is illustrated in figure \ref{figure:pok-layers}.
      The kernel executes the partitions, each partition contains its
      application code. Drivers are executed in partitions and don't reside
      inside the kernel.



      To build a such architecture, you must have:
      \begin{itemize}
         \item
            For each partition
            \begin{itemize}
               \item
                  The application code
               \item
                  The configuration code
            \end{itemize}
         \item
            For the kernel
            \begin{itemize}
               \item
                  The configuration code
            \end{itemize}
      \end{itemize}

      \onehugefig{imgs/partitions-build-steps}{Build steps for a partitioned
      system}{figure:partitioned-architecture-build}



      Then, each part of the system is compiled and integrated, as depicted in
      figure \ref{figure:partitioned-architecture-build}. The kernel is
      compiled and each partitions is compiled. Eeach part produces a binary
      file. Since POK relies on the ELF file format, each binary of each part is
      compiled into an ELF file.


      Then, we integrate \textbf{ALL} ELF files to produce a single bootable binary so
      that the final binary contains different binaries:
      the code for the kernel and the code of all partitions. Since POK relies
      on the ELF file format, the final ELF file contains other ELF files. The
      organization of the final binary is depiceted in figure
      \ref{figure:pok-elf-file}.


      When kernel boots, it loads each elf file of each partition in a different
      memory segment to achieve space isolation. So, each ELF file of each
      partition is loaded in a single and protected memory area of the system.

      \onehugefig{imgs/pok-elf-file}{ELF file format of a POK system}{figure:pok-elf-file}

      \subsection{Executive architecture}
      At this time, the execute architecture pattern is not finished.

\section{Kernel services}
   \subsection{Partitioning service}
   The partitioning service of POK isolates code in time and space. Each
   partition has one or more time slots to execute their code and they are
   isolated in a memory segment.

   Using this design guideline, one partition cannot access the memory of other
   partitions (and \textit{vice-versa}). During partitions initialization, POK
   automatically creates a memory segment for each partition and copy its code
   into this protected space.

   However, partitions can communicate with other partitions using so-called
   ports. Inter-partitions ports are also supervised by the kernel in order to
   avoid unallowed communication channel. See section
   \ref{kernel-service-communication} for more information.

   Partitions have time slots to execute their threads. During this execution
   time, they schedule their threads according to their own scheduling protocol
   so that partitions can schedule their threads in an independent way. This
   scheduling strategy is often described as a hierarchical scheduling.

   \subsection{Thread service}
   The thread service executes tasks. The system is built to execute a
   predefined number of tasks. When using partitioning services, each partitions
   has a predefined amount of tasks.

   The scheduler can be preemptive so that tasks can interrupt each other. The
   thread service can start, stop or pause a task (sleep).

   \subsection{Time service}
   The time service provides an efficient way to manage the time on your
   machine. It is used by the scheduler to scheduler partitions and tasks
   according to their timing requirements (period, execution time and so on).

   \subsection{Communication service}
   \label{kernel-service-communication}
   The kernel provides communication services. It allows partitions and threads
   to communicate. The communication service is achieved using \textit{ports}.
   \textit{Out} ports (ports that send data) can have several destinations 
   while \textit{in} ports (ports that receive data) can have only one source. 

   Data are sent and received on this ports. The kernel configuration specifies
   the owner of a port, its destination and its size.

   If you use partitioning service, each port is dedicated to a partition.
   Consequently, when creating the port, the kernel checks that requested port
   belongs to the partition.

      \subsubsection{Communication using network}
      When using the network, the port must be bound to a network interface so
      that data from/to the port will be sent over the network. The binding
      between a port and a network interface is specified in the kernel
      configuration.

      Please note that in POK, when you are using partitioning services, device
      drivers are executed in partitions.

   \subsection{Scheduling service}
   The scheduling service schedules tasks and partitions according to their
   timing requirements. It relies on the time service.

   Partitions are scheduled using a cyclic scheduling algorithm.

   Partitions threads are scheduled using a Round-Robin, RMS or other available
   scheduling algorithms.


\section{libpok services}
   \subsection{Thread management}
   Thread management consist in interfacing functions with the kernel. It
   provides functions to start/suspend/stop a thread. It provides also locking
   services for mutexes/semaphores and so on.

   \subsection{Communication service}
   Libpok provides two kind of communication services:
   \begin{itemize}
      \item
         \textbf{Inter-partition communication} which consists in
         kernel-interfacing functions to use kernel communication ports.
      \item
         \textbf{Intra-partition communication service} which provides
         communication facilities to communicate inside a partition.
   \end{itemize}

   In the following, we detail intra-partition communication services.

   Intra-partition communication service provides four communication patterns:
   \begin{enumerate}
      \item
         \textbf{Buffer} : thread send data. New data are queued according to a
         specific queueing policy. Items are dequeued when a task reads the
         buffer. We can store several instance of the same data.

         You need to define the \texttt{POK\_NEEDS\_BUFFERS} maccro to activate
         this service.
      \item
         \textbf{Blackboard} : a shared memory space to store a data. New
         instances of the data replace the older value. We can store only one
         instance of the same data.

         You need to define the \texttt{POK\_NEEDS\_BLACKBOARDS} maccro to activate
         this service.
      \item
         \textbf{Events} : are used to synchronized tasks. It corresponds to
         POSIX mutexes and conditions.

         You need to define the \texttt{POK\_NEEDS\_EVENTS} maccro to activate
         this service.
      \item
         \textbf{Semaphores} : counting semaphores, as in the POSIX standard.

         You need to define the \texttt{POK\_NEEDS\_SEMAPHORES} maccro to activate
         this service.
   \end{enumerate}

   \subsection{Memory allocator}
   POK also provides a memory allocator. This memory allocator was designed to
   be deterministic and highly configurable. You define the amount of memory for
   the memory allocator and the number of memory slices that can be allocated.

   Consequently, the memory allocator can be configured with different maccros.
   The service is activated by defining the
   \texttt{POK\_CONFIG\_NEEDS\_ALLOCATOR} maccro. Then, the
   \texttt{POK\_CONFIG\_ALLOCATOR\_MEMORY\_SIZE} is used to specify the amount
   of memory dedicated for the memory allocator. Finally the
   \texttt{POK\_CONFIG\_ALLOCATOR\_NB\_SPACES} specifies the number of spaces
   you can allocate with the memory allocator.

   This memory allocator can be used with the legacy layer (with the
   \texttt{pok\_allocator\_allocate()} or \texttt{pok\_allocator\_free()}
   functions) or with the C-library layer (\texttt{malloc()}, \texttt{free()},
   \texttt{calloc()}).

   \subsection{Mathematic library service}
   We also add mathematic functions to ease the portability of third-party code.
   These functions were imported from the NetBSD\footnote{http://www.netbsd.org}
   project. It provides all necessary functions to perform math operations
   (\texttt{sqrt()}, ...).

   To enable the libmath functions, you must define the maccro
   \texttt{POK\_NEEDS\_LIBMATH}.

   To have the complete list, please refer to the libpok reference manual
   available on each POK release. A list of these functions is also available in
   this document, in chapter \ref{chapter-api}.


   \subsection{Protocols}
   The libpok layer contains predefined protocols to marshall/unmarshall
   application data before sending them on the network. These protocols library
   could be used for several purposes: encrypt data before sending it on an
   unsecure network, adapt application data to constrained protocols such as
   CORBA, \ldots

   These protocols can be automatically used through \aadl models and
   appropriate properties associated to \aadl data ports on \aadl process
   components. To have more information about \aadl and protocol binding, see
   section \ref{chapter-automatic-configuration}.

   At this time, the libpok layer is focuses on crypto and provides the following protocols:
   \begin{itemize}
      \item
         Ceasar
      \item
         DES
      \item
         SSL
   \end{itemize}

   For each protocol, we have:
   \begin{itemize}
      \item
         A function to marshall data.
      \item
         A function to unmarshall data.
      \item
         An associated type if the protocol needs a special data type to store
         marshalled values.
   \end{itemize}

   Marshalling functions and types are described in their header files (see
   \texttt{des.h}, \texttt{ssl.h}, \texttt{ceasar.h} and so on). If there is no
   associated marshalling type, then, the marshall/unmarshall functions uses the
   same type as the application type or not particular type are required.


   Details of each protocol can be found in the API section (chapter
   \ref{chapter-api}).
